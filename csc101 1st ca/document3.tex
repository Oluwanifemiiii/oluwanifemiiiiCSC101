\documentclass{article}
\usepackage{amsmath}

\begin{document}
	\begin{itemize}
		\item \textbf{compound Interest} is interest paid on the original principal of a deposit or loan and also on the accumulated interest.
		\item The addition of interest to the principal is called \emph{compounding}
		\item Interest can be compounded periodically.
	\end{itemize}
		\begin{equation}
		A = P(1 + \frac{R}{n})^{nt}
		\end{equation}
		\begin{equation}
		I = A - P
		\end{equation}
	\begin{itemize}
		\item A= Accumulated Balance or amount
		\item P= Principal
		\item R= Annual rate (in decimal)
		\item n= Number of compounding periods per year.
		\item T= time 
		\item I= Interest
	\end{itemize}
	
\end{document}