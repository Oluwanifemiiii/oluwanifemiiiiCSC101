\documentclass{article}
\usepackage{amsmath}

\begin{document}
	\begin{itemize}
		\item When interest is compounded, the annual rate of interest (R) is called the \textbf{nominal rate}
		\item The \textbf{effective rate,R} is the simple interest rate that would yield the same amount of interest after 1 year.
		\item When a bank advertises a "7\% annual interest rate compounded daily and yielding 7.25\%", the nominal interest rate is 7\% and the effective rate is 7.25\%.
	\end{itemize}
\begin{equation}
R_{e} = [(1 + \frac{R}{n})^{n-1}]*100
\end{equation}
The effective rate is useful for comparing rates with different compounding periods.
\end{document}